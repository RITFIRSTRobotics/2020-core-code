% Document settings
\documentclass[11pt]{article}
\usepackage{fancyhdr}
\usepackage[margin=1in]{geometry}
\usepackage[pdftex]{graphicx}
\usepackage{multirow}
\usepackage{setspace}
\pagestyle{plain}
\usepackage{graphicx}
\usepackage{booktabs}
\usepackage{epstopdf}
%\usepackage{MnSymbol,wasysym}
\usepackage{amsmath}
%\usepackage{mathtools}
\usepackage{amssymb}
\usepackage{lipsum}
\usepackage{sympytex}
\usepackage{siunitx}
\setlength\parindent{0pt}
\graphicspath{{images/}{drawings/}}

% % % % % % % % % % % Header footer
% % % % % % % % % % %EDIT THIS % % % % % % % % % % % % % % % % % % % %
\pagestyle{fancy}
\fancyhf{}
\lhead{Imagine Project Control Protocol}
\rhead{Alex Kneipp}
\lfoot{RIT FIRST Alumni Association}
\cfoot{YYYY-MM-DD}
\rfoot{Page \thepage}
% % % % % % % % % % % % % % % % % % % % % % % % % % % % % % % % % % % % %


\begin{document}

%Setup the numbering scheme for equations and figures
\numberwithin{equation}{subsection}
\numberwithin{figure}{subsection}

\title{Imagine Project Control Protocol}
\author{Alex Kneipp}
\date{YYYY-MM-DD}
\maketitle

\hspace{1in}
%TODO Include an RIT FIRST logo
%\includegraphics[scale=0.9,trim=0cm 0in 0in 0.0in,clip]{RIT_KGCOE1}

\tableofcontents

\newpage

\section {Packet Types}
\paragraph{}
This standard defines several different packet types to better fit the packet structure to the type of data being sent.
Each packet type has a numerical value which is included in the packet header to encode the type.
The valid packet types, their encoded values, their intended purpose, 
and the network protocol which should be used to send them is shown in Table \ref{tab:pktTypes}.

\begin{table}[h]
    \centering
    \caption{Packet categories and their purpose}
    \label{tab:pktTypes}
    \begin{tabular}{|c|c|p{3in}|c|}
        \hline
        PacketType & Value & Purpose & Protocol \\ 
        \hline
        INIT & 0x00 & Robot Initialization & TCP \\ 
        \hline
        STATE\_REQUEST & 0x10 & A request for the recipient to respond with their current state. & TCP \\
        \hline
        STATE\_UPDATE & 0x12 & The current state of the sender. 
        Sent periodically or in response to a STATE\_REQUEST message. & TCP \\
        \hline
        CONFIG\_REQUEST & 0x20 & A request the that the recipient respond with their current configuration. & TCP \\
        \hline
        CONFIG\_RESPONSE & 0x21 & Sent in response to a CONFIG\_REQUEST message; 
            includes data about the senders configuration. & TCP \\
        \hline
        CONFIG\_UPDATE & 0x22 & A request for the recipient to update their configuration according to the data contained 
            in the message. & TCP \\
        \hline
        USER\_DATA & 0x30 & Data from the robot controllers & UDP \\
        \hline
        UPDATE\_STATUS & 0x40 & Optionally sent in response to CONFIG\_UPDATE or STATE\_UPDATE requests.
        Contains the result of the update.
        If the update failed, the message should contain a reason for the failure.
        If this message is not received in response to a *\_UPDATE message, the update should be assumed to be a success.& UDP \\
        \hline
        DEBUG & 0xFF & Debug data to be reported to a developer. & TCP/UDP \\
        \hline
    \end{tabular}
\end{table}

\section {Protocol Sequence}
\subsection{Initialization}
\paragraph{}
Upon boot up, a robot connects to the robot network and makes an INIT request containing its UUID to the FMS 
(The FMS network address must be configured in the robot configuration before startup.) 
The FMS then responds to the robot with an empty INIT message. 
The FMS then sends a CONFIG\_UDPATE message to the robot which configures it for a controller connected 
to the FMS and the alliance color corresponding to the controller it received. 
\paragraph{}
The CONFIG\_UPDATE message is not required to be sent immediately following the INIT response, 
in order to allow a human field manager to assign a controller and team to a robot which has just been added to the field.
An Initialization procedure may happen at any time, including during the match,
and the FMS should respond whenever it receives an INIT request.

\subsection{Match Start}
\paragraph{}
Before a robot may join a match, it must have successfully completed the entirety of the Initialization procedure 
(including the configuration).  
To start the match, the FMS sends a STATE\_UPDATE message to all of the robots it has successfully initialized 
which sets them to the ENABLED state.

\subsection {The Match}
\paragraph{}
During a match, the FMS should poll the connected ASCs for controller data, 
and then forward that data to the robots which it has fully initialized via USER\_DATA messages.
Each USER\_DATA message should be sent only to the robot for which the controller data is intended,
and it should only contain the data intended for that robot.
The FMS may optionally periodically send STATE\_REQUEST messages to each of the robots participating in a match,
and may act according to the response or lack thereof.

\subsection {Match End}
\paragraph{}
To stop a match regularly, the FMS sends a STATE\_UPDATE message to 
all of the robots it has successfully initialized to DISABLED.

\subsection{Emergency Stop}
\paragraph{}
To E-Stop a robot, the FMS sends a STATE\_UPDATE message to the target robot, which updates its state to E-STOPPED.
An E-STOPPED robot should stop all of its motors,
return to the default position, and ignore or respond with failure to all further messages until re-initialized.

\section {Packet Structures}

% If you need a bibliography for your document, add it here
%\begin{thebibliography}{9}
%\end{thebibliography}

\end{document}



