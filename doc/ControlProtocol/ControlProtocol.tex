% Document settings
\documentclass[11pt]{article}
\usepackage{fancyhdr}
\usepackage[margin=1in]{geometry}
\usepackage[pdftex]{graphicx}
\usepackage{multirow}
\usepackage{setspace}
\pagestyle{plain}
\usepackage{graphicx}
\usepackage{booktabs}
\usepackage{epstopdf}
%\usepackage{MnSymbol,wasysym}
\usepackage{amsmath}
%\usepackage{mathtools}
\usepackage{amssymb}
\usepackage{lipsum}
\usepackage{sympytex}
\usepackage{siunitx}
\setlength\parindent{0pt}
\graphicspath{{images/}{drawings/}}
\usepackage[acronym]{glossaries}

\usepackage{rotating}
\usepackage{array}
\usepackage[table]{xcolor}

% % % % % % % % % % % Header footer
% % % % % % % % % % %EDIT THIS % % % % % % % % % % % % % % % % % % % %
\pagestyle{fancy}
\fancyhf{}
\lhead{ImagineBots Control Protocol}
\rhead{Alex Kneipp}
\lfoot{RIT FIRST Alumni Association}
\cfoot{2020-01-18}
\rfoot{Page \thepage}
% % % % % % % % % % % % % % % % % % % % % % % % % % % % % % % % % % % % %
% % % % % % % % % % % %Glossary Setup % %% % % % %
\makeglossaries

\newacronym{tcp}{TCP}{Transmission Control Protocol}
\newacronym{udp}{UDP}{User Datagram Protocol}
\newacronym{uuid}{UUID}{Unique Universal ID}
\newacronym{fms}{FMS}{Field Management System}
\newacronym{asc}{ASC}{Arduino Scoring and Control}
% % % % % % % % % % % % % % % % % % % % % % % % % % % % % % % % % % % % %

\begin{document}

%Setup the numbering scheme for equations and figures
\numberwithin{equation}{subsection}
\numberwithin{figure}{subsection}

\title{ImagineBots Control Protocol}
\author{Alex Kneipp}
\date{2020-01-18}
\maketitle

\hspace{1in}
%TODO Include an RIT FIRST logo
%\includegraphics[scale=0.9,trim=0cm 0in 0in 0.0in,clip]{RIT_KGCOE1}

\tableofcontents


\newpage
\section {Packet Types}
\paragraph{}
This standard defines several different packet types to better fit the packet structure to the type of data being sent.
Each packet type has a numerical value which is included in the packet header to encode the type.
The valid packet types, their encoded values, their intended purpose, 
and the network protocol which should be used to send them is shown in Table \ref{tab:pktTypes}.

\begin{table}[h!]
    \centering
    \caption{Packet categories and their purpose}
    \label{tab:pktTypes}
    \begin{tabular}{|c|c|p{3in}|c|}
        \hline
        PacketType & Value & Purpose & Protocol \\ 
        \hline
        INIT & 0x00 & Robot Initialization & \acrshort{tcp} \\ 
        \hline
        STATE\_REQUEST & 0x10 & A request for the recipient to respond with their current state. & \acrshort{tcp} \\
        \hline
        STATE\_RESPONSE & 0x11 & The current state of the sender. Sent periodically or in response to a STATE\_REQUEST message. & \acrshort{tcp} \\
        \hline
        STATE\_UPDATE & 0x12 & A request for the recipient to update their state according to the data contained in the message. & \acrshort{tcp} \\
        \hline
        CONFIG\_REQUEST & 0x20 & A request for the recipient responds with their current configuration. & \acrshort{tcp} \\
        \hline
        CONFIG\_RESPONSE & 0x21 & Sent in response to a CONFIG\_REQUEST message; 
            includes data about the senders configuration. & \acrshort{tcp} \\
        \hline
        CONFIG\_UPDATE & 0x22 & A request for the recipient to update their configuration according to the data contained 
            in the message. & \acrshort{tcp} \\
        \hline
        USER\_DATA & 0x30 & Data from the robot controllers & \acrshort{udp} \\
        \hline
        UPDATE\_STATUS & 0x40 & Optionally sent in response to CONFIG\_UPDATE or STATE\_UPDATE requests.
        Contains the result of the update.  If the update failed, the message should contain a reason for the failure.
        If this message is not received in response to a *\_UPDATE message, the update should be assumed to be a success.
        & \acrshort{udp} \\
        \hline
        DEBUG & 0xFF & Debug data to be reported to a developer. & \acrshort{tcp}/\acrshort{udp} \\
        \hline
    \end{tabular}
\end{table}

\section {Protocol Sequence}
\subsection{Initialization}
\paragraph{}
Upon boot up, a robot connects to the robot network and makes an INIT request containing its \acrfull{uuid} to the \acrfull{fms} 
(The \acrshort{fms} network address must be configured in the robot configuration before startup.) 
The \acrshort{fms} then responds to the robot with an INIT message with a \acrshort{uuid} of all zeros. 
The robot should synchronize it's program time to the timestamp in the INIT message from the \acrshort{fms}.
The \acrshort{fms} then sends a CONFIG\_UDPATE message to the robot which configures it for a controller connected 
to the \acrshort{fms} and the alliance color corresponding to the controller it received. 
\paragraph{}
The CONFIG\_UPDATE message is not required to be sent immediately following the INIT response, 
in order to allow a human field manager to assign a controller and team to a robot which has just been added to the field.
An Initialization procedure may happen at any time, including during the match,
and the \acrshort{fms} should respond whenever it receives an INIT request.

\subsection{Match Start}
\paragraph{}
Before a robot may join a match, it must have successfully completed the entirety of the Initialization procedure 
(including the configuration).  
To start the match, the \acrshort{fms} sends a STATE\_UPDATE message to all of the robots it has successfully initialized 
which sets them to the ENABLED state.

\subsection {The Match}
\paragraph{}
During a match, the \acrshort{fms} should poll the connected \acrfull{asc} for controller data, 
and then forward that data to the robots which it has fully initialized via USER\_DATA messages.
Each USER\_DATA message should be sent only to the robot for which the controller data is intended,
and it should only contain the data intended for that robot.
The \acrshort{fms} may optionally periodically send STATE\_REQUEST messages to each of the robots participating in a match,
and may act according to the response or lack thereof.

\subsection {Match End}
\paragraph{}
To stop a match regularly, the \acrshort{fms} sends a STATE\_UPDATE message to 
all of the robots it has successfully initialized to DISABLED.

\subsection{Emergency Stop}
\paragraph{}
To emergency stop (e-stop) a robot, the \acrshort{fms} sends a STATE\_UPDATE message to the target robot, which updates its state to E-STOPPED.
This packet can be sent at any time after a robot is initialized. An E-STOPPED robot should turn off all of its motors 
and ignore or respond with failure to all further messages until re-initialized.

\section {Packet Structures}
\paragraph{}
The robot control packets generally follow a Type-Length-Value (TLV) structure. Packets may be sent via either UDP or TCP (depending on the type of the packet), and robots should implement listeners for both protocols.

\subsection {Packet Header}
\paragraph{}
Each packet contains a header with the message type (which determines how the robot should unpack the data in the packet),
and message length, and a timestamp.
\newline
\begin{table}[h!]
    \centering
    \caption{Structure of the header of all packets}
    \label{tab:headerDef}
    \begin{tabular}{|p{1cm}|m{0.04cm}|m{0.04cm}|m{0.04cm}|m{0.04cm}|m{0.04cm}|m{0.04cm}|m{0.04cm}|m{0.04cm}|m{0.04cm}|
        m{0.04cm}|m{0.04cm}|m{0.04cm}|m{0.04cm}|m{0.04cm}|m{0.04cm}|m{0.04cm}|m{0.04cm}|m{0.04cm}|m{0.04cm}|m{0.04cm}|
        m{0.04cm}|m{0.04cm}|m{0.04cm}|m{0.04cm}|m{0.04cm}|m{0.04cm}|m{0.04cm}|m{0.04cm}|m{0.04cm}|m{0.04cm}|m{0.04cm}|m{0.04cm}|}
        \hline
        Word & 
        \begin{sideways}31\end{sideways} &
        \begin{sideways}30\end{sideways} & 
        \begin{sideways}29\end{sideways} &
        \begin{sideways}28\end{sideways} &
        \begin{sideways}27\end{sideways} &
        \begin{sideways}26\end{sideways} &
        \begin{sideways}25\end{sideways} &
        \begin{sideways}24\end{sideways} &
        \begin{sideways}23\end{sideways} &
        \begin{sideways}22\end{sideways} &
        \begin{sideways}21\end{sideways} &
        \begin{sideways}20\end{sideways} &
        \begin{sideways}19\end{sideways} &
        \begin{sideways}18\end{sideways} &
        \begin{sideways}17\end{sideways} &
        \begin{sideways}16\end{sideways} &
        \begin{sideways}15\end{sideways} &
        \begin{sideways}14\end{sideways} &
        \begin{sideways}13\end{sideways} &
        \begin{sideways}12\end{sideways} &
        \begin{sideways}11\end{sideways} &
        \begin{sideways}10\end{sideways} &
        \begin{sideways}9\end{sideways} &
        \begin{sideways}8\end{sideways} &
        \begin{sideways}7\end{sideways} &
        \begin{sideways}6\end{sideways} &
        \begin{sideways}5\end{sideways} &
        \begin{sideways}4\end{sideways} &
        \begin{sideways}3\end{sideways} &
        \begin{sideways}2\end{sideways} &
        \begin{sideways}1\end{sideways} &
        \begin{sideways}0\end{sideways} \\
        \hline
        0 & \multicolumn{8}{c|}{Message Type} & \multicolumn{24}{c|}{Message Length} \\
        \hline
        1 & \multicolumn{32}{c|}{Timestamp} \\
        \hline
        2..n & \multicolumn{32}{c|}{Data} \\
        \hline
    \end{tabular}
\end{table}
Message Type: Denotes the type of message contained in this packet, see Table 1 for more information.\\
Message Length: The length of the message, including the header, in bytes.
Timestamp: The time, in milliseconds, since the \acrshort{fms} has started, stored as a 32 bit unsigned integer.
Robots should update their timestamp to match that of the \acrshort{fms} when they recieve their initialization packet.\\
Data: The data which this packet carries, which must be $Message Length - 2$ words long.\\

\subsection {INIT Packets}
\paragraph{}
Initialization packets are sent as part of the initialization handshake between a starting up robot and the FMS.  
As such it contains information necessary to uniquely identify a robot.  
\newline
\begin{table}[h!]
    \centering
    \caption{Structure of INIT packets}
    \label{tab:InitDef}
    \begin{tabular}{|p{1cm}|m{0.04cm}|m{0.04cm}|m{0.04cm}|m{0.04cm}|m{0.04cm}|m{0.04cm}|m{0.04cm}|m{0.04cm}|m{0.04cm}|
        m{0.04cm}|m{0.04cm}|m{0.04cm}|m{0.04cm}|m{0.04cm}|m{0.04cm}|m{0.04cm}|m{0.04cm}|m{0.04cm}|m{0.04cm}|m{0.04cm}|
        m{0.04cm}|m{0.04cm}|m{0.04cm}|m{0.04cm}|m{0.04cm}|m{0.04cm}|m{0.04cm}|m{0.04cm}|m{0.04cm}|m{0.04cm}|m{0.04cm}|m{0.04cm}|}
        \hline
        Word & 
        \begin{sideways}31\end{sideways} &
        \begin{sideways}30\end{sideways} & 
        \begin{sideways}29\end{sideways} &
        \begin{sideways}28\end{sideways} &
        \begin{sideways}27\end{sideways} &
        \begin{sideways}26\end{sideways} &
        \begin{sideways}25\end{sideways} &
        \begin{sideways}24\end{sideways} &
        \begin{sideways}23\end{sideways} &
        \begin{sideways}22\end{sideways} &
        \begin{sideways}21\end{sideways} &
        \begin{sideways}20\end{sideways} &
        \begin{sideways}19\end{sideways} &
        \begin{sideways}18\end{sideways} &
        \begin{sideways}17\end{sideways} &
        \begin{sideways}16\end{sideways} &
        \begin{sideways}15\end{sideways} &
        \begin{sideways}14\end{sideways} &
        \begin{sideways}13\end{sideways} &
        \begin{sideways}12\end{sideways} &
        \begin{sideways}11\end{sideways} &
        \begin{sideways}10\end{sideways} &
        \begin{sideways}9\end{sideways} &
        \begin{sideways}8\end{sideways} &
        \begin{sideways}7\end{sideways} &
        \begin{sideways}6\end{sideways} &
        \begin{sideways}5\end{sideways} &
        \begin{sideways}4\end{sideways} &
        \begin{sideways}3\end{sideways} &
        \begin{sideways}2\end{sideways} &
        \begin{sideways}1\end{sideways} &
        \begin{sideways}0\end{sideways} \\
        \hline
        [0,1] & \multicolumn{32}{c|}{Header}\\
        \hline
        2 & \multicolumn{32}{c|}{Robot UUID} \\
        \hline
    \end{tabular}
\end{table}

Robot UUID: A unique identifier for the robot which MAY NOT conflict with any other robots on the field.

\subsection {STATE\_REQUEST Packets}
\paragraph{}
State request packets are empty except for the standard header with the appropriate information.
\newline
\begin{table}[h!]
    \centering
    \caption{Structure of STATE\_REQUEST packets}
    \label{tab:stateReqestDef}
    \begin{tabular}{|p{1cm}|m{0.04cm}|m{0.04cm}|m{0.04cm}|m{0.04cm}|m{0.04cm}|m{0.04cm}|m{0.04cm}|m{0.04cm}|m{0.04cm}|
        m{0.04cm}|m{0.04cm}|m{0.04cm}|m{0.04cm}|m{0.04cm}|m{0.04cm}|m{0.04cm}|m{0.04cm}|m{0.04cm}|m{0.04cm}|m{0.04cm}|
        m{0.04cm}|m{0.04cm}|m{0.04cm}|m{0.04cm}|m{0.04cm}|m{0.04cm}|m{0.04cm}|m{0.04cm}|m{0.04cm}|m{0.04cm}|m{0.04cm}|m{0.04cm}|}
        \hline
        Word & 
        \begin{sideways}31\end{sideways} &
        \begin{sideways}30\end{sideways} & 
        \begin{sideways}29\end{sideways} &
        \begin{sideways}28\end{sideways} &
        \begin{sideways}27\end{sideways} &
        \begin{sideways}26\end{sideways} &
        \begin{sideways}25\end{sideways} &
        \begin{sideways}24\end{sideways} &
        \begin{sideways}23\end{sideways} &
        \begin{sideways}22\end{sideways} &
        \begin{sideways}21\end{sideways} &
        \begin{sideways}20\end{sideways} &
        \begin{sideways}19\end{sideways} &
        \begin{sideways}18\end{sideways} &
        \begin{sideways}17\end{sideways} &
        \begin{sideways}16\end{sideways} &
        \begin{sideways}15\end{sideways} &
        \begin{sideways}14\end{sideways} &
        \begin{sideways}13\end{sideways} &
        \begin{sideways}12\end{sideways} &
        \begin{sideways}11\end{sideways} &
        \begin{sideways}10\end{sideways} &
        \begin{sideways}9\end{sideways} &
        \begin{sideways}8\end{sideways} &
        \begin{sideways}7\end{sideways} &
        \begin{sideways}6\end{sideways} &
        \begin{sideways}5\end{sideways} &
        \begin{sideways}4\end{sideways} &
        \begin{sideways}3\end{sideways} &
        \begin{sideways}2\end{sideways} &
        \begin{sideways}1\end{sideways} &
        \begin{sideways}0\end{sideways} \\
        \hline
        [0,1] & \multicolumn{32}{c|}{Header}\\
        \hline
    \end{tabular}
\end{table}

\subsection {STATE\_RESPONSE Packets}
\paragraph{}
These packets contatin arbitrary data about the state of the message sender.
\newline
\begin{table}[h!]
    \centering
    \caption{Structure of STATE\_RESPONSE packets}
    \label{tab:stateResponseDef}
    \begin{tabular}{|p{1cm}|m{0.04cm}|m{0.04cm}|m{0.04cm}|m{0.04cm}|m{0.04cm}|m{0.04cm}|m{0.04cm}|m{0.04cm}|m{0.04cm}|
        m{0.04cm}|m{0.04cm}|m{0.04cm}|m{0.04cm}|m{0.04cm}|m{0.04cm}|m{0.04cm}|m{0.04cm}|m{0.04cm}|m{0.04cm}|m{0.04cm}|
        m{0.04cm}|m{0.04cm}|m{0.04cm}|m{0.04cm}|m{0.04cm}|m{0.04cm}|m{0.04cm}|m{0.04cm}|m{0.04cm}|m{0.04cm}|m{0.04cm}|m{0.04cm}|}
        \hline
        Word & 
        \begin{sideways}31\end{sideways} &
        \begin{sideways}30\end{sideways} & 
        \begin{sideways}29\end{sideways} &
        \begin{sideways}28\end{sideways} &
        \begin{sideways}27\end{sideways} &
        \begin{sideways}26\end{sideways} &
        \begin{sideways}25\end{sideways} &
        \begin{sideways}24\end{sideways} &
        \begin{sideways}23\end{sideways} &
        \begin{sideways}22\end{sideways} &
        \begin{sideways}21\end{sideways} &
        \begin{sideways}20\end{sideways} &
        \begin{sideways}19\end{sideways} &
        \begin{sideways}18\end{sideways} &
        \begin{sideways}17\end{sideways} &
        \begin{sideways}16\end{sideways} &
        \begin{sideways}15\end{sideways} &
        \begin{sideways}14\end{sideways} &
        \begin{sideways}13\end{sideways} &
        \begin{sideways}12\end{sideways} &
        \begin{sideways}11\end{sideways} &
        \begin{sideways}10\end{sideways} &
        \begin{sideways}9\end{sideways} &
        \begin{sideways}8\end{sideways} &
        \begin{sideways}7\end{sideways} &
        \begin{sideways}6\end{sideways} &
        \begin{sideways}5\end{sideways} &
        \begin{sideways}4\end{sideways} &
        \begin{sideways}3\end{sideways} &
        \begin{sideways}2\end{sideways} &
        \begin{sideways}1\end{sideways} &
        \begin{sideways}0\end{sideways} \\
        \hline
        [0,1] & \multicolumn{32}{c|}{Header}\\
        \hline
        2 & \multicolumn{8}{c|}{State Code} & \multicolumn{24}{c|}{\cellcolor[gray]{.6}Reserved} \\
        \hline
        3..n & \multicolumn{32}{c|}{Arbitrary Data} \\
        \hline
    \end{tabular}
\end{table}

State Code: An integer value signifying the state of the sender.  Valid codes are shown in Table \ref{tab:StateCodes}.

%TODO a lot of these don't describe how the \acrshort{fms} handles states this table says it's allowed to be in.
%That should be fixed
\begin{table}[h!]
    \centering
    \caption{Robot and \acrshort{fms} States and Codes}
    \label{tab:StateCodes}
    \begin{tabular}{|c|p{1cm}|p{3in}|c|}
        \hline
        State Name & Code & Meaning & \acrshort{fms} or Robot? \\
        \hline
        INITIALIZED & 0x00 & The robot has been initialized but not configured (See the Initialization sequence) & Robot \\
        \hline
        DISABLED & 0x01 & The robot is disabled and should not take action based on or be sent any USER\_DATA packets. & Robot/FMS\\
        \hline
        ENABLED & 0x02 & The robot is enabled and should react to USER\_DATA packets. & Robot/\acrshort{fms} \\
        \hline
        E-STOPPED & 0xFF & The robot has been E-STOPPED and should not respond to any further packets except with error messages. & Robot/\acrshort{fms} \\
        \hline
    \end{tabular}
\end{table}

Arbitrary Data: User-defined data which may be used or ignored by either the robot or FMS.

\subsection {STATE\_UPDATE Packets}
\paragraph{}
These packets are sent when the \acrshort{fms} wishes to update the state of a robot.
\newline
\begin{table}[h!]
    \centering
    \caption{Structure of STATE\_UPDATE packets}
    \label{tab:stateUpdateDef}
    \begin{tabular}{|p{1cm}|m{0.04cm}|m{0.04cm}|m{0.04cm}|m{0.04cm}|m{0.04cm}|m{0.04cm}|m{0.04cm}|m{0.04cm}|m{0.04cm}|
        m{0.04cm}|m{0.04cm}|m{0.04cm}|m{0.04cm}|m{0.04cm}|m{0.04cm}|m{0.04cm}|m{0.04cm}|m{0.04cm}|m{0.04cm}|m{0.04cm}|
        m{0.04cm}|m{0.04cm}|m{0.04cm}|m{0.04cm}|m{0.04cm}|m{0.04cm}|m{0.04cm}|m{0.04cm}|m{0.04cm}|m{0.04cm}|m{0.04cm}|m{0.04cm}|}
        \hline
        Word & 
        \begin{sideways}31\end{sideways} &
        \begin{sideways}30\end{sideways} & 
        \begin{sideways}29\end{sideways} &
        \begin{sideways}28\end{sideways} &
        \begin{sideways}27\end{sideways} &
        \begin{sideways}26\end{sideways} &
        \begin{sideways}25\end{sideways} &
        \begin{sideways}24\end{sideways} &
        \begin{sideways}23\end{sideways} &
        \begin{sideways}22\end{sideways} &
        \begin{sideways}21\end{sideways} &
        \begin{sideways}20\end{sideways} &
        \begin{sideways}19\end{sideways} &
        \begin{sideways}18\end{sideways} &
        \begin{sideways}17\end{sideways} &
        \begin{sideways}16\end{sideways} &
        \begin{sideways}15\end{sideways} &
        \begin{sideways}14\end{sideways} &
        \begin{sideways}13\end{sideways} &
        \begin{sideways}12\end{sideways} &
        \begin{sideways}11\end{sideways} &
        \begin{sideways}10\end{sideways} &
        \begin{sideways}9\end{sideways} &
        \begin{sideways}8\end{sideways} &
        \begin{sideways}7\end{sideways} &
        \begin{sideways}6\end{sideways} &
        \begin{sideways}5\end{sideways} &
        \begin{sideways}4\end{sideways} &
        \begin{sideways}3\end{sideways} &
        \begin{sideways}2\end{sideways} &
        \begin{sideways}1\end{sideways} &
        \begin{sideways}0\end{sideways} \\
        \hline
        [0,1] & \multicolumn{32}{c|}{Header}\\
        \hline
        2 & \multicolumn{8}{c|}{New State Code} & \multicolumn{24}{c|}{\cellcolor[gray]{.6}Reserved} \\
        \hline
        3..n & \multicolumn{32}{c|}{Arbitrary Data} \\
        \hline
    \end{tabular}
\end{table}

New State Code: An integer value signifying the state the robot or FMS should change to.
Valid codes are shown in Table \ref{tab:StateCodes}\\
Arbitrary Data: User-defined data which may be used or ignored by either the robot or FMS.

\subsection {CONFIG\_REQUEST Packets}
\paragraph{}
These packets are sent by any agent to request a config value or values of another.
\newline
\begin{table}[h!]
    \centering
    \caption{Structure of CONFIG\_REQUEST packets}
    \label{tab:configRequestDef}
    \begin{tabular}{|p{1cm}|m{0.04cm}|m{0.04cm}|m{0.04cm}|m{0.04cm}|m{0.04cm}|m{0.04cm}|m{0.04cm}|m{0.04cm}|m{0.04cm}|
        m{0.04cm}|m{0.04cm}|m{0.04cm}|m{0.04cm}|m{0.04cm}|m{0.04cm}|m{0.04cm}|m{0.04cm}|m{0.04cm}|m{0.04cm}|m{0.04cm}|
        m{0.04cm}|m{0.04cm}|m{0.04cm}|m{0.04cm}|m{0.04cm}|m{0.04cm}|m{0.04cm}|m{0.04cm}|m{0.04cm}|m{0.04cm}|m{0.04cm}|m{0.04cm}|}
        \hline
        Word & 
        \begin{sideways}31\end{sideways} &
        \begin{sideways}30\end{sideways} & 
        \begin{sideways}29\end{sideways} &
        \begin{sideways}28\end{sideways} &
        \begin{sideways}27\end{sideways} &
        \begin{sideways}26\end{sideways} &
        \begin{sideways}25\end{sideways} &
        \begin{sideways}24\end{sideways} &
        \begin{sideways}23\end{sideways} &
        \begin{sideways}22\end{sideways} &
        \begin{sideways}21\end{sideways} &
        \begin{sideways}20\end{sideways} &
        \begin{sideways}19\end{sideways} &
        \begin{sideways}18\end{sideways} &
        \begin{sideways}17\end{sideways} &
        \begin{sideways}16\end{sideways} &
        \begin{sideways}15\end{sideways} &
        \begin{sideways}14\end{sideways} &
        \begin{sideways}13\end{sideways} &
        \begin{sideways}12\end{sideways} &
        \begin{sideways}11\end{sideways} &
        \begin{sideways}10\end{sideways} &
        \begin{sideways}9\end{sideways} &
        \begin{sideways}8\end{sideways} &
        \begin{sideways}7\end{sideways} &
        \begin{sideways}6\end{sideways} &
        \begin{sideways}5\end{sideways} &
        \begin{sideways}4\end{sideways} &
        \begin{sideways}3\end{sideways} &
        \begin{sideways}2\end{sideways} &
        \begin{sideways}1\end{sideways} &
        \begin{sideways}0\end{sideways} \\
        \hline
        [0,1] & \multicolumn{32}{c|}{Header}\\
        \hline
        2 & \multicolumn{32}{c|}{\multirow{2}{*}{Keys List}} \\
        \cline{1-1}
        ... & \multicolumn{32}{c|}{}\\
        \hline
        n & \multicolumn{16}{c|}{Keys List (cont)} & \multicolumn{16}{c|}{\cellcolor[gray]{.6}Unused} \\
        \hline
    \end{tabular}
\end{table}
\newline
Keys List:  A sequence of nul-terminated strings which represent the configuration keys of the receiving agent.
It may be the special keyword “ALL”, which tells the receiving agent to respond with their entire configuration.\\
Unused: Padding bytes to make the packet aligned to a 32-bit boundary

\subsection {CONFIG\_RESPONSE Packets}
\paragraph{}
These packets are sent in response to a CONFIG\_REQUEST packet and contains the reqeusted configuration keys.
\newline
\begin{table}[h!]
    \centering
    \caption{Structure of CONFIG\_RESPONSE packets}
    \label{tab:configResponseDef}
    \begin{tabular}{|p{1cm}|m{0.04cm}|m{0.04cm}|m{0.04cm}|m{0.04cm}|m{0.04cm}|m{0.04cm}|m{0.04cm}|m{0.04cm}|m{0.04cm}|
        m{0.04cm}|m{0.04cm}|m{0.04cm}|m{0.04cm}|m{0.04cm}|m{0.04cm}|m{0.04cm}|m{0.04cm}|m{0.04cm}|m{0.04cm}|m{0.04cm}|
        m{0.04cm}|m{0.04cm}|m{0.04cm}|m{0.04cm}|m{0.04cm}|m{0.04cm}|m{0.04cm}|m{0.04cm}|m{0.04cm}|m{0.04cm}|m{0.04cm}|m{0.04cm}|}
        \hline
        Word & 
        \begin{sideways}31\end{sideways} &
        \begin{sideways}30\end{sideways} & 
        \begin{sideways}29\end{sideways} &
        \begin{sideways}28\end{sideways} &
        \begin{sideways}27\end{sideways} &
        \begin{sideways}26\end{sideways} &
        \begin{sideways}25\end{sideways} &
        \begin{sideways}24\end{sideways} &
        \begin{sideways}23\end{sideways} &
        \begin{sideways}22\end{sideways} &
        \begin{sideways}21\end{sideways} &
        \begin{sideways}20\end{sideways} &
        \begin{sideways}19\end{sideways} &
        \begin{sideways}18\end{sideways} &
        \begin{sideways}17\end{sideways} &
        \begin{sideways}16\end{sideways} &
        \begin{sideways}15\end{sideways} &
        \begin{sideways}14\end{sideways} &
        \begin{sideways}13\end{sideways} &
        \begin{sideways}12\end{sideways} &
        \begin{sideways}11\end{sideways} &
        \begin{sideways}10\end{sideways} &
        \begin{sideways}9\end{sideways} &
        \begin{sideways}8\end{sideways} &
        \begin{sideways}7\end{sideways} &
        \begin{sideways}6\end{sideways} &
        \begin{sideways}5\end{sideways} &
        \begin{sideways}4\end{sideways} &
        \begin{sideways}3\end{sideways} &
        \begin{sideways}2\end{sideways} &
        \begin{sideways}1\end{sideways} &
        \begin{sideways}0\end{sideways} \\
        \hline
        [0,1] & \multicolumn{32}{c|}{Header}\\
        \hline
        2 & \multicolumn{32}{c|}{\multirow{2}{*}{Key-value List}} \\
        \cline{1-1}
        ... & \multicolumn{32}{c|}{}\\
        \hline
        n & \multicolumn{16}{c|}{Key-value List (cont)} & \multicolumn{16}{c|}{\cellcolor[gray]{.6}Unused} \\
        \hline
    \end{tabular}
\end{table}
\newline
Key-value list: A semi-colon (;) separated list of key-value pairs representing the configuration values to 
update or create on the reciever.
Each key is a nul-terminated string immediately followed by a TLV structure (defined in Table \ref{tab:TLVDef}) signifying the value of the configuration value.
Valid TLV types are defined in Table \ref{tab:configTypes}.
Unused: Padding bytes to force packet alignment to a 32-bit word.

\begin{table}[h!]
    \centering
    \caption{Valid Configuration value types}
    \label{tab:configTypes}
    \begin{tabular}{|p{3in}|p{.5in}|}
        \hline
        Type & Type Code\\
        \hline
        Integer & 0x00\\
        \hline
        Float & 0x01\\
        \hline
        Double Precision Floating Point & 0x02\\
        \hline
        C-String & 0x03\\
        \hline
        Boolean & 0x04 \\
        \hline
    \end{tabular}
\end{table}

\subsection {CONFIG\_UPDATE Packets}
\paragraph{}
An agent sends this packet to attempt to change the configuration of another agent.
\newline
\begin{table}[h!]
    \centering
    \caption{Structure of CONFIG\_UPDATE packets}
    \label{tab:configUpdateDef}
    \begin{tabular}{|p{1cm}|m{0.04cm}|m{0.04cm}|m{0.04cm}|m{0.04cm}|m{0.04cm}|m{0.04cm}|m{0.04cm}|m{0.04cm}|m{0.04cm}|
        m{0.04cm}|m{0.04cm}|m{0.04cm}|m{0.04cm}|m{0.04cm}|m{0.04cm}|m{0.04cm}|m{0.04cm}|m{0.04cm}|m{0.04cm}|m{0.04cm}|
        m{0.04cm}|m{0.04cm}|m{0.04cm}|m{0.04cm}|m{0.04cm}|m{0.04cm}|m{0.04cm}|m{0.04cm}|m{0.04cm}|m{0.04cm}|m{0.04cm}|m{0.04cm}|}
        \hline
        Word & 
        \begin{sideways}31\end{sideways} &
        \begin{sideways}30\end{sideways} & 
        \begin{sideways}29\end{sideways} &
        \begin{sideways}28\end{sideways} &
        \begin{sideways}27\end{sideways} &
        \begin{sideways}26\end{sideways} &
        \begin{sideways}25\end{sideways} &
        \begin{sideways}24\end{sideways} &
        \begin{sideways}23\end{sideways} &
        \begin{sideways}22\end{sideways} &
        \begin{sideways}21\end{sideways} &
        \begin{sideways}20\end{sideways} &
        \begin{sideways}19\end{sideways} &
        \begin{sideways}18\end{sideways} &
        \begin{sideways}17\end{sideways} &
        \begin{sideways}16\end{sideways} &
        \begin{sideways}15\end{sideways} &
        \begin{sideways}14\end{sideways} &
        \begin{sideways}13\end{sideways} &
        \begin{sideways}12\end{sideways} &
        \begin{sideways}11\end{sideways} &
        \begin{sideways}10\end{sideways} &
        \begin{sideways}9\end{sideways} &
        \begin{sideways}8\end{sideways} &
        \begin{sideways}7\end{sideways} &
        \begin{sideways}6\end{sideways} &
        \begin{sideways}5\end{sideways} &
        \begin{sideways}4\end{sideways} &
        \begin{sideways}3\end{sideways} &
        \begin{sideways}2\end{sideways} &
        \begin{sideways}1\end{sideways} &
        \begin{sideways}0\end{sideways} \\
        \hline
        [0,1] & \multicolumn{32}{c|}{Header}\\
        \hline
        2 & \multicolumn{32}{c|}{\multirow{2}{*}{Key-value List}} \\
        \cline{1-1}
        ... & \multicolumn{32}{c|}{}\\
        \hline
        n & \multicolumn{16}{c|}{Keys-value List (cont)} & \multicolumn{16}{c|}{\cellcolor[gray]{.6}Unused} \\
        \hline
    \end{tabular}
\end{table}
\newline
Key-value list: A semi-colon (;) separated list of key-value pairs representing the configuration values to 
update or create on the reciever.
Each key is a nul-terminated string immediately followed by a TLV structure signifying the value of the configuration value.
Valid types for the TLV structure are defined in table \ref{tab:configTypes}.
Unused:  Padding bytes to force packet alignment to a 32-bit word.

\subsection {USER\_DATA Packets}
\paragraph{}
USER\_DATA Packets contain data from the human controller.
\begin{table}[h!]
    \centering
    \caption{Structure of USER\_DATA packets}
    \label{tab:userDataDef}
    \begin{tabular}{|p{1cm}|p{0.04cm}|p{0.04cm}|m{0.04cm}|m{0.04cm}|m{0.04cm}|m{0.04cm}|m{0.04cm}|m{0.04cm}|m{0.04cm}|
        m{0.04cm}|m{0.04cm}|m{0.04cm}|m{0.04cm}|m{0.04cm}|m{0.04cm}|m{0.04cm}|m{0.04cm}|m{0.04cm}|m{0.04cm}|m{0.04cm}|
        m{0.04cm}|m{0.04cm}|m{0.04cm}|m{0.04cm}|m{0.04cm}|m{0.04cm}|m{0.04cm}|m{0.04cm}|m{0.04cm}|m{0.04cm}|m{0.04cm}|m{0.04cm}|}
        \hline
        Word & 
        \begin{sideways}31\end{sideways} &
        \begin{sideways}30\end{sideways} & 
        \begin{sideways}29\end{sideways} &
        \begin{sideways}28\end{sideways} &
        \begin{sideways}27\end{sideways} &
        \begin{sideways}26\end{sideways} &
        \begin{sideways}25\end{sideways} &
        \begin{sideways}24\end{sideways} &
        \begin{sideways}23\end{sideways} &
        \begin{sideways}22\end{sideways} &
        \begin{sideways}21\end{sideways} &
        \begin{sideways}20\end{sideways} &
        \begin{sideways}19\end{sideways} &
        \begin{sideways}18\end{sideways} &
        \begin{sideways}17\end{sideways} &
        \begin{sideways}16\end{sideways} &
        \begin{sideways}15\end{sideways} &
        \begin{sideways}14\end{sideways} &
        \begin{sideways}13\end{sideways} &
        \begin{sideways}12\end{sideways} &
        \begin{sideways}11\end{sideways} &
        \begin{sideways}10\end{sideways} &
        \begin{sideways}9\end{sideways} &
        \begin{sideways}8\end{sideways} &
        \begin{sideways}7\end{sideways} &
        \begin{sideways}6\end{sideways} &
        \begin{sideways}5\end{sideways} &
        \begin{sideways}4\end{sideways} &
        \begin{sideways}3\end{sideways} &
        \begin{sideways}2\end{sideways} &
        \begin{sideways}1\end{sideways} &
        \begin{sideways}0\end{sideways} \\
        \hline
        [0,1] & \multicolumn{32}{c|}{Header}\\
        \hline
        2 & \multicolumn{8}{c|}{Left Joystick X} & \multicolumn{8}{c|}{Left Joystick Y} & 
            \multicolumn{8}{c|}{Right Joystick X} & \multicolumn{8}{c|}{Right Joystick Y} \\
        \hline
        3 &A&B& \multicolumn{14}{c|}{\cellcolor[gray]{.6} Reserved} & \multicolumn{16}{c|}{Controller UUID} \\
        \hline
    \end{tabular}
\end{table}
\newline
Left Joystick X: The X axis of the left joystick on the controller.  An unsigned integer value from 0 to 255.\\
Left Joystick Y: The Y axis of the left joystick on the controller.  An unsigned integer value from 0 to 255.\\
Right Joystick X: The X axis of the right joystick on the controller.  An unsigned integer value from 0 to 255.\\
Right Joystick Y: The Y axis of the right joystick on the controller.  An unsigned integer value from 0 to 255.\\
A: The current state of button “A” (which button?).  0 is unpressed, 1 is pressed.\\
B: The current state of button “B” (which button?).  0 is unpressed, 1 is pressed.\\
Controller UID: A universally unique ID for this controller.

\subsection {UPDATE\_STATUS Packets}
\paragraph{}
UPDATE\_STATUS packets are sent in response to a CONFIG\_UPDATE or STATE\_UPDATE packet, 
and contain whether or not the update command was successful.
\begin{table}[h!]
    \centering
    \caption{Structure of UPDATE\_STATUS packets}
    \label{tab:updateStatusDef}
    \begin{tabular}{|p{1cm}|p{0.04cm}|p{0.04cm}|m{0.04cm}|m{0.04cm}|m{0.04cm}|m{0.04cm}|m{0.04cm}|m{0.04cm}|m{0.04cm}|
        m{0.04cm}|m{0.04cm}|m{0.04cm}|m{0.04cm}|m{0.04cm}|m{0.04cm}|m{0.04cm}|m{0.04cm}|m{0.04cm}|m{0.04cm}|m{0.04cm}|
        m{0.04cm}|m{0.04cm}|m{0.04cm}|m{0.04cm}|m{0.04cm}|m{0.04cm}|m{0.04cm}|m{0.04cm}|m{0.04cm}|m{0.04cm}|m{0.04cm}|m{0.04cm}|}
        \hline
        Word & 
        \begin{sideways}31\end{sideways} &
        \begin{sideways}30\end{sideways} & 
        \begin{sideways}29\end{sideways} &
        \begin{sideways}28\end{sideways} &
        \begin{sideways}27\end{sideways} &
        \begin{sideways}26\end{sideways} &
        \begin{sideways}25\end{sideways} &
        \begin{sideways}24\end{sideways} &
        \begin{sideways}23\end{sideways} &
        \begin{sideways}22\end{sideways} &
        \begin{sideways}21\end{sideways} &
        \begin{sideways}20\end{sideways} &
        \begin{sideways}19\end{sideways} &
        \begin{sideways}18\end{sideways} &
        \begin{sideways}17\end{sideways} &
        \begin{sideways}16\end{sideways} &
        \begin{sideways}15\end{sideways} &
        \begin{sideways}14\end{sideways} &
        \begin{sideways}13\end{sideways} &
        \begin{sideways}12\end{sideways} &
        \begin{sideways}11\end{sideways} &
        \begin{sideways}10\end{sideways} &
        \begin{sideways}9\end{sideways} &
        \begin{sideways}8\end{sideways} &
        \begin{sideways}7\end{sideways} &
        \begin{sideways}6\end{sideways} &
        \begin{sideways}5\end{sideways} &
        \begin{sideways}4\end{sideways} &
        \begin{sideways}3\end{sideways} &
        \begin{sideways}2\end{sideways} &
        \begin{sideways}1\end{sideways} &
        \begin{sideways}0\end{sideways} \\
        \hline
        [0,1] & \multicolumn{32}{c|}{Header}\\
        \hline
        2 & \multicolumn{8}{c|}{Status Code} & \multicolumn{24}{c|}{\cellcolor[gray]{.6} Reserved} \\
        \hline
    \end{tabular}
\end{table}

Status Code: The result of the last received UPDATE\_* type packet.
Table \ref{tab:updateStatusCodes} summarizes valid codes.

\begin{table}[h!]
    \centering
    \caption{Update status codes and descriptions}
    \label{tab:updateStatusCodes}
    \begin{tabular}{|p{2in}|p{.8in}|p{3.25in}|}
        \hline
        Status Name & Status Code & Status Description \\
        \hline
        SUCCESS & 0x00 & The update was accepted and implemented in the receiving entity. \\
        \hline
        INVALID\_STATE & 0x01 & The requested state change was for an unknown state \\
        \hline
        PERMISSION\_DENIED & 0x02 & The sending entity does not have permission to update 
        the state or configuration of the receiving entity. \\
        \hline
    \end{tabular}
\end{table}

\subsection{DEBUG Packets}
\begin{table}[h!]
    \centering
    \caption{Structure of DEBUG packets}
    \label{tab:debugDef}
    \begin{tabular}{|p{1cm}|p{0.04cm}|p{0.04cm}|m{0.04cm}|m{0.04cm}|m{0.04cm}|m{0.04cm}|m{0.04cm}|m{0.04cm}|m{0.04cm}|
        m{0.04cm}|m{0.04cm}|m{0.04cm}|m{0.04cm}|m{0.04cm}|m{0.04cm}|m{0.04cm}|m{0.04cm}|m{0.04cm}|m{0.04cm}|m{0.04cm}|
        m{0.04cm}|m{0.04cm}|m{0.04cm}|m{0.04cm}|m{0.04cm}|m{0.04cm}|m{0.04cm}|m{0.04cm}|m{0.04cm}|m{0.04cm}|m{0.04cm}|m{0.04cm}|}
        \hline
        Word & 
        \begin{sideways}31\end{sideways} &
        \begin{sideways}30\end{sideways} & 
        \begin{sideways}29\end{sideways} &
        \begin{sideways}28\end{sideways} &
        \begin{sideways}27\end{sideways} &
        \begin{sideways}26\end{sideways} &
        \begin{sideways}25\end{sideways} &
        \begin{sideways}24\end{sideways} &
        \begin{sideways}23\end{sideways} &
        \begin{sideways}22\end{sideways} &
        \begin{sideways}21\end{sideways} &
        \begin{sideways}20\end{sideways} &
        \begin{sideways}19\end{sideways} &
        \begin{sideways}18\end{sideways} &
        \begin{sideways}17\end{sideways} &
        \begin{sideways}16\end{sideways} &
        \begin{sideways}15\end{sideways} &
        \begin{sideways}14\end{sideways} &
        \begin{sideways}13\end{sideways} &
        \begin{sideways}12\end{sideways} &
        \begin{sideways}11\end{sideways} &
        \begin{sideways}10\end{sideways} &
        \begin{sideways}9\end{sideways} &
        \begin{sideways}8\end{sideways} &
        \begin{sideways}7\end{sideways} &
        \begin{sideways}6\end{sideways} &
        \begin{sideways}5\end{sideways} &
        \begin{sideways}4\end{sideways} &
        \begin{sideways}3\end{sideways} &
        \begin{sideways}2\end{sideways} &
        \begin{sideways}1\end{sideways} &
        \begin{sideways}0\end{sideways} \\
        \hline
        [0,1] & \multicolumn{32}{c|}{Header}\\
        \hline
        2 & \multicolumn{4}{c|}{Code Status} & \multicolumn{28}{c|}{Git Commit Hash}\\
        \hline
        3 & \multicolumn{32}{c|}{Robot UUID}\\
        \hline
        4 & \multicolumn{8}{c|}{Robot State} & \multicolumn{8}{c|}{\cellcolor[gray]{.6}Reserved}&
            \multicolumn{16}{c|}{Number of Config Entries}\\
        \hline
        ... & \multicolumn{32}{c|}{Arbitrary Data}\\
        \hline
        n & \multicolumn{16}{c|}{Arbitrary Data (Cont)} & \multicolumn{16}{c|}{\cellcolor[gray]{.6}Unused}\\
        \hline
    \end{tabular}
\end{table}
Code Status: 4-bit value denoting the state of the code relative to the current git commit.
See Table \ref{tab:CodeStatus} for a list of valid values.  This should be inserted into the code as a constant at compile time.\\
Git Commit Hash: The hash of the current git commit of the FMS or Robot code (not core code) running on the sender.
The hash should be inserted into the code at compile time (i.e. using a makefile)\\
Robot UUID: A unique identifier for the robot which MAY NOT conflict with any other robots on the field.\\
Robot State: The current state of the robot (see the *\_STATE message definitions for a more complete description)\\
Number of Config Entries: The current number of configuration entries the robot has\\
Arbitrary Data: Whatever data the programmer sees fit to send in a debug message to the reciever.\\
Unused:  Unused padding bits to align the message to a 32 bit word.\\

\begin{table}[h!]
    \centering
    \caption{Valid values of the Code Status field in DEBUG Packets}.
    \label{tab:CodeStatus}
    \begin{tabular}{|p{2in}|p{.8in}|p{3.25in}|}
        \hline
        Name & Value & Description \\
        \hline
        Up to date & 0x0 & The code running on the robot is the same as the commit with the provided hash \\
        \hline
        Ahead & 0x1 & There are changes in the running code that are not in the git commit\\
        \hline
    \end{tabular}
\end{table}

\subsection {TLV Structure}
\paragraph{}
Several packets contain general purpose TLV (Type, Length, Value) structures.
The general structure of them is defined in Table \ref{tab:TLVDef};
each of the packets individually define the valid types for their data.
\newline

\begin{table}[h!]
    \centering
    \caption{Structure of TLV data structures}
    \label{tab:TLVDef}
    \begin{tabular}{|p{1cm}|p{0.04cm}|p{0.04cm}|m{0.04cm}|m{0.04cm}|m{0.04cm}|m{0.04cm}|m{0.04cm}|m{0.04cm}|m{0.04cm}|
        m{0.04cm}|m{0.04cm}|m{0.04cm}|m{0.04cm}|m{0.04cm}|m{0.04cm}|m{0.04cm}|m{0.04cm}|m{0.04cm}|m{0.04cm}|m{0.04cm}|
        m{0.04cm}|m{0.04cm}|m{0.04cm}|m{0.04cm}|m{0.04cm}|m{0.04cm}|m{0.04cm}|m{0.04cm}|m{0.04cm}|m{0.04cm}|m{0.04cm}|m{0.04cm}|}
        \hline
        Word & 
        \begin{sideways}31\end{sideways} &
        \begin{sideways}30\end{sideways} & 
        \begin{sideways}29\end{sideways} &
        \begin{sideways}28\end{sideways} &
        \begin{sideways}27\end{sideways} &
        \begin{sideways}26\end{sideways} &
        \begin{sideways}25\end{sideways} &
        \begin{sideways}24\end{sideways} &
        \begin{sideways}23\end{sideways} &
        \begin{sideways}22\end{sideways} &
        \begin{sideways}21\end{sideways} &
        \begin{sideways}20\end{sideways} &
        \begin{sideways}19\end{sideways} &
        \begin{sideways}18\end{sideways} &
        \begin{sideways}17\end{sideways} &
        \begin{sideways}16\end{sideways} &
        \begin{sideways}15\end{sideways} &
        \begin{sideways}14\end{sideways} &
        \begin{sideways}13\end{sideways} &
        \begin{sideways}12\end{sideways} &
        \begin{sideways}11\end{sideways} &
        \begin{sideways}10\end{sideways} &
        \begin{sideways}9\end{sideways} &
        \begin{sideways}8\end{sideways} &
        \begin{sideways}7\end{sideways} &
        \begin{sideways}6\end{sideways} &
        \begin{sideways}5\end{sideways} &
        \begin{sideways}4\end{sideways} &
        \begin{sideways}3\end{sideways} &
        \begin{sideways}2\end{sideways} &
        \begin{sideways}1\end{sideways} &
        \begin{sideways}0\end{sideways} \\
        \hline
        [0,1] & \multicolumn{32}{c|}{Header}\\
        \hline
        ... & \multicolumn{32}{c|}{\cellcolor[gray]{.6}Miscellaneous Data}\\
        \hline
        n & \multicolumn{8}{c|}{Value Type} & \multicolumn{24}{c|}{Value Length}\\
        \hline
        n+1 & \multicolumn{32}{c|}{Value Data}\\
        \hline
        n+m & \multicolumn{16}{c|}{Value Data (Cont.)} & \multicolumn{16}{c|}{\cellcolor[gray]{.6}Misc. Data}\\
        \hline
    \end{tabular}
\end{table}


Value Type: Type of the data in this structure, defined on a per-application basis. \\
Value Length: The length of the entire structure in bytes.  Stored in an unsigned integer. \\
Value Data: The value stored in this TLV. TLVs need not be aligned to a 32-bit word, but should be aligned to the byte. \\

\clearpage

\section{Failure Modes}
\subsection{Robot Disconnection}
\paragraph{}
The \acrshort{fms} should regularly (about every 5s) send STATE\_REQUEST packets to each robot it has Initialized 
in order to detect if the robot has been disconnected.  If a robot has been disconnected, 
it must be reinitialized by the \acrshort{fms}.
As such, robots should respond to a STATE\_REQUEST packet as soon as possible, and should monitor incoming STATE\_UPDATE packets
so they can also tell if they have been disconnected.  This functionality may be disabled by a value in the \acrshort{fms}
configuration, which must be sent over in the CONFIG\_UPDATE packet as part of the initialization sequence.

\paragraph{}
If disconnect detection is disabled and a robot is rebooted, it will send an INIT packet to the \acrshort{fms}.
In order to correctly handle this situation, the \acrshort{fms} must keep the \acrshort{uuid} of all robots, 
and if it sees an INIT packet from a robot it has already initialized, 
it should re-initialize the robot with the same paramaters the robot already had, rather
than generating a new set of paramaters.

\subsection{Emergency Stop}
\paragraph{}
If the robot or entire field has been emergency stopped by a user, the e-stopped component must be fully rebooted.
After rebooting, the Robot Disconnection failure mode should be followed if the component was a robot.  
If the field was stopped, each robot should also be rebooted and the field should be started as usual.

\clearpage

\printglossary[type=\acronymtype]
% If you need a bibliography for your document, add it here
%\begin{thebibliography}{9}
%\end{thebibliography}

\end{document}
